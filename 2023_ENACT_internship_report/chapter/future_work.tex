%\section{Future Work}

In response to \textbf{RQ2.3}, our next steps will involve integrating the selected code refactoring techniques such as \texttt{Convert Local Variable to Field}, \texttt{Introduce Parameter Object}, and \texttt{Move Method} into the identified \texttt{StatementEdit} class within Gin. Before this integration, we need to integrate code smell detection technique in the Gin tool. This will allow us for efficient detection and tracing of code smells. 

\vspace{.5em}
Our first step is to identify if there are any code smells present. Only upon detecting code smells can we then integrate the selected code refactoring techniques. Under this step, we have two sub-steps. The first is to translate the code into Abstract Syntax Tree (AST) format. The second sub-step is to identify the tool that detects code smells from the code in Abstract Syntax Tree (AST) format using specific code smell detection technique. Once identified, the techniques used by the identified code smell detection tool will be integrated into the 'Gin' tool for detecting code smells.

\vspace{.5em}
\cite{liu2017detection} presents famous code smeel detection tools: \texttt{Checkstyle}, \texttt{JDeodorant}. These two tool detect code smells from the code in Abstract Syntax Tree (AST) format using specific code smell detection technique. \texttt{Checkstyle} is a well-known static code analysis tool, it can detect 4 code smells Large Class, Long Method, Long Parameter List, and Duplicated Code.\texttt{JDeodorant} is an Eclipse plug-in\cite{DBLP:conf/wcre/TsantalisCC18}. It can automatically detect 4 code smells Feature Envy, God Class, Long Method, and Switch Statement. It can achieve high detection accuracy. In addition, JDeodorant can achieve a good visualization of detection results. But at present it only can detect 4 code smells. \texttt{Checkstyle} uses a metric-based approach to detect code smells, where it checks the source code for violations of specified metrics such as cyclomatic complexity, lines of code, and number of parameters\cite{DBLP:journals/ese/FontanaMZM16}. It was not mentioned specifically what approach used in the JDeodorant code smell detection tool.

\vspace{.5em}
\cite{DBLP:journals/corr/abs-2012-08842} presents an up-to-date review of state-of-the-art techniques and tools used for code smell detection and visualization. The study found that the most commonly used approaches for code smell detection are search-based (30.1\%), metric-based (24.1\%), and symptom-based (19.3\%). The study does offer insights into the strengths and limitations of various approaches, which can guide us in deciding the most suitable technique for our specific needs. For instance, search-based approaches are effective in detecting code smells. The effectiveness of a given technique could depend on factors like the specific codebase under analysis, the programming language in use, and the desired accuracy and precision for detecting code smells.

\vspace{.5em}
We have selected the search-based code smell detection technique for integration into the Gin tool. Search-based code smell detection is a technique that uses search algorithms to identify instances of code smells in software systems. To integrate search-based code smell detection into GIN, we would need to modify the Gin tool’s source code to incorporate the desired search-based technique. This would likely involve implementing the search algorithm and defining a fitness function that evaluates candidate solutions based on their ability to detect code smells. We may also need to modify GIN’s existing genetic operators (e.g., mutation and crossover) to work with the new search-based approach.

\vspace{.5em}
After integrating code smell detection techniques into the GIN tool, we can proceed to incorporate code refactoring techniques. Once these are integrated, we will conduct experiments using the extended version of the GIN tool to determine whether the optimized version of the program or project results in energy savings in the software domain thereby addressing Research Question \textbf{RQ2.3}.