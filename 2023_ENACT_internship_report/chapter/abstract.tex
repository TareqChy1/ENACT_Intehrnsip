This internship report provides a comprehensive overview of the research undertaken to explore tactics for enhancing software energy efficiency. We have chosen code refactoring as a tactic to improve energy efficiency in software. Subsequently, we utilized the genetic improvement(gin) tool to obtain an optimized version of the code. We then examined whether this optimized version results in a reduction in energy consumption. Our findings confirmed that the optimized program does indeed consume less energy. Our next step is to integrate code refactoring techniques with the GIN tool. Finally, we will conduct experiments to determine if the integrated gin tool can indeed bring about a significant reduction in energy consumption. The study was conducted under the supervision of Sophie Chabridon from  Télécom SudParis/SAMOVAR Lab and Denisse Muñante Arzapalo from ENSIIE/ SAMOVAR Lab Évry, France. The research methodology involved a combination of experimental analysis and practical experiments.\par

\vspace{5pt}
The initial phase of the research involved an extensive literature review, which provided the foundation for the study. Existing approaches, techniques, and technologies related to software energy efficiency were analyzed, enabling an understanding of the subject matter.\par

\vspace{5pt}
The internship experience was greatly beneficial, and I would like to express gratitude to the internship supervisors for their unwavering support throughout the entire process. Despite their busy schedules, the supervisors demonstrated attentiveness to my needs and facilitated a seamless integration into the team. Their guidance, advice, and assistance contributed significantly to the success of the internship, creating a positive and productive atmosphere.\par

\vspace{5pt}
Furthermore, I would like to extend appreciation to the Télécom SudParis/SAMOVAR Lab, E4C, for providing the opportunity to conduct the internship within their research environment. The collaborative and stimulating atmosphere of the lab enhanced the learning experience and fostered meaningful contributions to the field of software energy efficiency.\par

\vspace{5pt}
Additionally, I acknowledge the contributions of Professor Maxime Lefrançois, Professor Piere Maret, and all the teachers at Ecole des Mines de Saint-Étienne and Jean Monnet University. Their dedication and investment in my academic pursuits have been instrumental in shaping the research skills and knowledge required for this internship. I express gratitude for their ongoing support and mentorship.\par

%\vspace{5em}
%\hfill Tareq Md Rabiul Hossain Chy % Align text to the right side

